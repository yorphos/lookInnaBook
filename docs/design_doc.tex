\documentclass{article}
\usepackage{graphicx}
\usepackage{pdflscape}
\usepackage{lmodern}
\usepackage[T1]{fontenc}
\usepackage{textcomp}
\usepackage{underscore}
\usepackage{listofitems}
\graphicspath{./}

\newcommand{\schema}[2]{#1(#2)}
\newcommand{\pkey}[1]{%
  \setsepchar{,}%
  \readlist*\pkeylist{#1}%
  \foreachitem\x\in\pkeylist[]{\ifnum\xcnt=1\else, \fi\underline{\x}}%
}
\newcommand{\fkey}[1]{%
  \setsepchar{,}%
  \readlist*\fkeylist{#1}%
  \foreachitem\x\in\fkeylist[]{\ifnum\xcnt=1\else, \fi\textit{\x}}%
}

\newcommand{\trightarrow}{\(\rightarrow\)}

\title{COMP3005 Project Design Document}
\author{Steven Pham}

\begin{document}
\maketitle
\section{ER Diagram}
The ER diagram is shown on the next page in landscape format.
Some assumptions made are:
\begin{itemize}
  \item We only offer services to Canadian customers.
  \item Payment can be processed with only the card number, expiry, CVV (number on back), and the cardholder name
  \item Each order can only ship to one address
  \item Each book can be uniquely identified by an ISBN
  \item A credit card number does not uniquely identify a method of payment (I did some cursory research and apparently sometimes these are reused with different CVV numbers)
\end{itemize}

\begin{landscape}
\includegraphics[height=\textwidth]{er}
\end{landscape}

\section{Relations Schema}
The above ER diagram can be broken down into the relation schema below. Primary keys are denoted by underscores. Foreign keys are denoted in italics.
\begin{itemize}
  \item \schema{book}{\pkey{isbn}, title, author_name, genre, publisher, num_pages, price, author_royalties, reorder_threshold, \fkey{publisher_id}}
  \item \schema{address}{\pkey{address_id}, street_address, postal_code, province}
  \item \schema{customer}{\pkey{customer_id}, name, email, password_hash, password_salt, \fkey{default_shipping_address_id, default_payment_info_id}}
  \item \schema{payment_info}{\pkey{payment_info_id}, name_on_card, expiry, card_number, cvv, \fkey{billing_address}}
  \item \schema{publisher}{\pkey{publisher_id}, company_name, phone_number, bank_information, \fkey{address_id}}
  \item \schema{order}{\pkey{order_id}, \fkey{customer_id, shipping_address}, tracking_number, order_status, order_date, \fkey{payment_info_id}}
  \item \schema{in_order}{\fkey{\pkey{isbn, order_id}}, quantity}
  \item \schema{restock_order}{\pkey{restock_order_id}, \fkey{isbn}, quantity, price_per_unit, order_date, order_status}
  \item \schema{in_cart}{\fkey{\pkey{isbn, customer_id}}, quantity}
  \item \schema{owner}{\pkey{owner_id}, name, email, password_hash, password_salt}
  \item \schema{book_collection}{\pkey{collection_id}, \fkey{curator_owner_id}}
  \item \schema{in_collection}{\fkey{\pkey{collection_id, isbn}}}
\end{itemize}

\section{Functional Dependencies}
Below are the functional dependencies for this domain.
\begin{itemize}
  \item ISBN \trightarrow{} Title, AuthorName, Genre, Publisher, NumPages, Price, AuthorRoyalties, ReorderThreshold
  \item AddressID \trightarrow{} StreetAddress, PostalCode, Province
  \item CustomerID \trightarrow{} CustomerName, CustomerEmail, CustomerPasswordHash, CustomerPasswordSalt, DefaultShippingAddressID, DefaultPaymentInfoID
  \item PaymentInfoID \trightarrow{} NameOnCard, ExpiryDate, CardNumber, CVV, BillingAddressID
  \item PublisherID \trightarrow{} CompanyName, PhoneNumber, BankInformation, AddressID
  \item OrderID \trightarrow{} CustomerID, TrackingNum, OrderStatus, OrderDate, ShippingAddressID, PaymentInfoID
  \item OrderID, BookISBN \trightarrow{} OrderQuantity
  \item CustomerID, BookISBN \trightarrow{} CartQuantity
  \item RestockOrderID \trightarrow{} BookISBN, Quantity, PricePerUnit, OrderDate, OrderStatus
  \item OwnerID \trightarrow{} OwnerName, OwnerEmail, OwnerPasswordHash, OwnerPasswordSalt
  \item BookCollectionID \trightarrow{} OwnerID
\end{itemize}

\section{Testing For Good Form}
We will now test to make sure that all of our relations are in good form. We will test that each relation is in 3NF.
\subsection{Book}
Functional dependencies:
\begin{itemize}
  \item ISBN \trightarrow{} Title, AuthorName, Genre, Publisher, NumPages, Price, AuthorRoyalties, ReorderThreshold
\end{itemize}

ISBN is trivially a super key so this relation is in BCNF.

\subsection{AddressID}
Functional dependencies:
\begin{itemize}
  \item AddressID \trightarrow{} StreetAddress, PostalCode, Province
\end{itemize}

AddressID is trivially a super key for address.

\subsection{Customer, PaymentInfo, Publisher, Order, RestockOrder, Owner, BookCollection}
All of these relations are also 3NF in a similar way where there is only one functional dependency which is some ID attribute to the rest of the relation.

\subsection{InOrder}
Functional dependencies:
\begin{itemize}
  \item OrderID, BookISBN \trightarrow{} OrderQuantity
\end{itemize}

(OrderID, BookISBN) is trivially the super key since it determines the other attribute in the relation.

\subsection{InCart}
Functional dependencies:
\begin{itemize}
  \item CustomerID, BookISBN \trightarrow{} CartQuantity
\end{itemize}

(CustomerID, BookISBN) is trivially the super key since it determines the other attribute in the relation.

All of our relations are in good form.


\end{document}
